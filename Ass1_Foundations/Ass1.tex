% Options for packages loaded elsewhere
\PassOptionsToPackage{unicode}{hyperref}
\PassOptionsToPackage{hyphens}{url}
%
\documentclass[
]{article}
\usepackage{amsmath,amssymb}
\usepackage{iftex}
\ifPDFTeX
  \usepackage[T1]{fontenc}
  \usepackage[utf8]{inputenc}
  \usepackage{textcomp} % provide euro and other symbols
\else % if luatex or xetex
  \usepackage{unicode-math} % this also loads fontspec
  \defaultfontfeatures{Scale=MatchLowercase}
  \defaultfontfeatures[\rmfamily]{Ligatures=TeX,Scale=1}
\fi
\usepackage{lmodern}
\ifPDFTeX\else
  % xetex/luatex font selection
\fi
% Use upquote if available, for straight quotes in verbatim environments
\IfFileExists{upquote.sty}{\usepackage{upquote}}{}
\IfFileExists{microtype.sty}{% use microtype if available
  \usepackage[]{microtype}
  \UseMicrotypeSet[protrusion]{basicmath} % disable protrusion for tt fonts
}{}
\makeatletter
\@ifundefined{KOMAClassName}{% if non-KOMA class
  \IfFileExists{parskip.sty}{%
    \usepackage{parskip}
  }{% else
    \setlength{\parindent}{0pt}
    \setlength{\parskip}{6pt plus 2pt minus 1pt}}
}{% if KOMA class
  \KOMAoptions{parskip=half}}
\makeatother
\usepackage{xcolor}
\usepackage[margin=1in]{geometry}
\usepackage{color}
\usepackage{fancyvrb}
\newcommand{\VerbBar}{|}
\newcommand{\VERB}{\Verb[commandchars=\\\{\}]}
\DefineVerbatimEnvironment{Highlighting}{Verbatim}{commandchars=\\\{\}}
% Add ',fontsize=\small' for more characters per line
\usepackage{framed}
\definecolor{shadecolor}{RGB}{248,248,248}
\newenvironment{Shaded}{\begin{snugshade}}{\end{snugshade}}
\newcommand{\AlertTok}[1]{\textcolor[rgb]{0.94,0.16,0.16}{#1}}
\newcommand{\AnnotationTok}[1]{\textcolor[rgb]{0.56,0.35,0.01}{\textbf{\textit{#1}}}}
\newcommand{\AttributeTok}[1]{\textcolor[rgb]{0.13,0.29,0.53}{#1}}
\newcommand{\BaseNTok}[1]{\textcolor[rgb]{0.00,0.00,0.81}{#1}}
\newcommand{\BuiltInTok}[1]{#1}
\newcommand{\CharTok}[1]{\textcolor[rgb]{0.31,0.60,0.02}{#1}}
\newcommand{\CommentTok}[1]{\textcolor[rgb]{0.56,0.35,0.01}{\textit{#1}}}
\newcommand{\CommentVarTok}[1]{\textcolor[rgb]{0.56,0.35,0.01}{\textbf{\textit{#1}}}}
\newcommand{\ConstantTok}[1]{\textcolor[rgb]{0.56,0.35,0.01}{#1}}
\newcommand{\ControlFlowTok}[1]{\textcolor[rgb]{0.13,0.29,0.53}{\textbf{#1}}}
\newcommand{\DataTypeTok}[1]{\textcolor[rgb]{0.13,0.29,0.53}{#1}}
\newcommand{\DecValTok}[1]{\textcolor[rgb]{0.00,0.00,0.81}{#1}}
\newcommand{\DocumentationTok}[1]{\textcolor[rgb]{0.56,0.35,0.01}{\textbf{\textit{#1}}}}
\newcommand{\ErrorTok}[1]{\textcolor[rgb]{0.64,0.00,0.00}{\textbf{#1}}}
\newcommand{\ExtensionTok}[1]{#1}
\newcommand{\FloatTok}[1]{\textcolor[rgb]{0.00,0.00,0.81}{#1}}
\newcommand{\FunctionTok}[1]{\textcolor[rgb]{0.13,0.29,0.53}{\textbf{#1}}}
\newcommand{\ImportTok}[1]{#1}
\newcommand{\InformationTok}[1]{\textcolor[rgb]{0.56,0.35,0.01}{\textbf{\textit{#1}}}}
\newcommand{\KeywordTok}[1]{\textcolor[rgb]{0.13,0.29,0.53}{\textbf{#1}}}
\newcommand{\NormalTok}[1]{#1}
\newcommand{\OperatorTok}[1]{\textcolor[rgb]{0.81,0.36,0.00}{\textbf{#1}}}
\newcommand{\OtherTok}[1]{\textcolor[rgb]{0.56,0.35,0.01}{#1}}
\newcommand{\PreprocessorTok}[1]{\textcolor[rgb]{0.56,0.35,0.01}{\textit{#1}}}
\newcommand{\RegionMarkerTok}[1]{#1}
\newcommand{\SpecialCharTok}[1]{\textcolor[rgb]{0.81,0.36,0.00}{\textbf{#1}}}
\newcommand{\SpecialStringTok}[1]{\textcolor[rgb]{0.31,0.60,0.02}{#1}}
\newcommand{\StringTok}[1]{\textcolor[rgb]{0.31,0.60,0.02}{#1}}
\newcommand{\VariableTok}[1]{\textcolor[rgb]{0.00,0.00,0.00}{#1}}
\newcommand{\VerbatimStringTok}[1]{\textcolor[rgb]{0.31,0.60,0.02}{#1}}
\newcommand{\WarningTok}[1]{\textcolor[rgb]{0.56,0.35,0.01}{\textbf{\textit{#1}}}}
\usepackage{longtable,booktabs,array}
\usepackage{calc} % for calculating minipage widths
% Correct order of tables after \paragraph or \subparagraph
\usepackage{etoolbox}
\makeatletter
\patchcmd\longtable{\par}{\if@noskipsec\mbox{}\fi\par}{}{}
\makeatother
% Allow footnotes in longtable head/foot
\IfFileExists{footnotehyper.sty}{\usepackage{footnotehyper}}{\usepackage{footnote}}
\makesavenoteenv{longtable}
\usepackage{graphicx}
\makeatletter
\def\maxwidth{\ifdim\Gin@nat@width>\linewidth\linewidth\else\Gin@nat@width\fi}
\def\maxheight{\ifdim\Gin@nat@height>\textheight\textheight\else\Gin@nat@height\fi}
\makeatother
% Scale images if necessary, so that they will not overflow the page
% margins by default, and it is still possible to overwrite the defaults
% using explicit options in \includegraphics[width, height, ...]{}
\setkeys{Gin}{width=\maxwidth,height=\maxheight,keepaspectratio}
% Set default figure placement to htbp
\makeatletter
\def\fps@figure{htbp}
\makeatother
\setlength{\emergencystretch}{3em} % prevent overfull lines
\providecommand{\tightlist}{%
  \setlength{\itemsep}{0pt}\setlength{\parskip}{0pt}}
\setcounter{secnumdepth}{-\maxdimen} % remove section numbering
\ifLuaTeX
  \usepackage{selnolig}  % disable illegal ligatures
\fi
\usepackage{bookmark}
\IfFileExists{xurl.sty}{\usepackage{xurl}}{} % add URL line breaks if available
\urlstyle{same}
\hypersetup{
  pdftitle={DAT320: Compulsory assignment 1},
  pdfauthor={Group 4},
  hidelinks,
  pdfcreator={LaTeX via pandoc}}

\title{DAT320: Compulsory assignment 1}
\author{Group 4}
\date{2024-09-16}

\begin{document}
\maketitle

\begin{Shaded}
\begin{Highlighting}[]
\FunctionTok{options}\NormalTok{(}\AttributeTok{contrasts =} \FunctionTok{c}\NormalTok{(}\StringTok{"contr.sum"}\NormalTok{, }\StringTok{"contr.poly"}\NormalTok{))}
\FunctionTok{require}\NormalTok{(}\StringTok{"ggplot2"}\NormalTok{)}
\FunctionTok{require}\NormalTok{(}\StringTok{"dplyr"}\NormalTok{)}
\FunctionTok{require}\NormalTok{(}\StringTok{"ppcor"}\NormalTok{)}
\FunctionTok{require}\NormalTok{(}\StringTok{"caret"}\NormalTok{)}
\FunctionTok{require}\NormalTok{(}\StringTok{"tidyr"}\NormalTok{)}
\end{Highlighting}
\end{Shaded}

\section{Exercise 1 - R syntax \& data
structures}\label{exercise-1---r-syntax-data-structures}

\subsection{Task a}\label{task-a}

\begin{Shaded}
\begin{Highlighting}[]
\NormalTok{gapminder }\OtherTok{\textless{}{-}} \FunctionTok{read.csv}\NormalTok{(}\StringTok{"gapminder.csv"}\NormalTok{)}

\FunctionTok{summary}\NormalTok{(gapminder)}
\end{Highlighting}
\end{Shaded}

\begin{verbatim}
##        X            country           continent              year     
##  Min.   :   1.0   Length:1704        Length:1704        Min.   :1952  
##  1st Qu.: 426.8   Class :character   Class :character   1st Qu.:1966  
##  Median : 852.5   Mode  :character   Mode  :character   Median :1980  
##  Mean   : 852.5                                         Mean   :1980  
##  3rd Qu.:1278.2                                         3rd Qu.:1993  
##  Max.   :1704.0                                         Max.   :2007  
##     lifeExp           pop              gdpPercap       
##  Min.   :23.60   Min.   :6.001e+04   Min.   :   241.2  
##  1st Qu.:48.20   1st Qu.:2.794e+06   1st Qu.:  1202.1  
##  Median :60.71   Median :7.024e+06   Median :  3531.8  
##  Mean   :59.47   Mean   :2.960e+07   Mean   :  7215.3  
##  3rd Qu.:70.85   3rd Qu.:1.959e+07   3rd Qu.:  9325.5  
##  Max.   :82.60   Max.   :1.319e+09   Max.   :113523.1
\end{verbatim}

\subsection{Task b}\label{task-b}

\begin{Shaded}
\begin{Highlighting}[]
\NormalTok{gapminder }\SpecialCharTok{\%\textgreater{}\%}
  \FunctionTok{group\_by}\NormalTok{(country) }\SpecialCharTok{\%\textgreater{}\%}
  \FunctionTok{ggplot}\NormalTok{(}\FunctionTok{aes}\NormalTok{(}\AttributeTok{x=}\NormalTok{year, }\AttributeTok{y=}\NormalTok{lifeExp, }\AttributeTok{gorup=}\NormalTok{country,  }\AttributeTok{colour=}\NormalTok{continent)) }\SpecialCharTok{+}
    \FunctionTok{geom\_line}\NormalTok{()}
\end{Highlighting}
\end{Shaded}

\includegraphics{Ass1_files/figure-latex/E1_b-1.pdf}

\subsection{Task c}\label{task-c}

\begin{Shaded}
\begin{Highlighting}[]
\NormalTok{knitr}\SpecialCharTok{::}\FunctionTok{kable}\NormalTok{(}
\NormalTok{  gapminder }\SpecialCharTok{\%\textgreater{}\%}
    \FunctionTok{group\_by}\NormalTok{(continent, year) }\SpecialCharTok{\%\textgreater{}\%}
    \FunctionTok{summarise\_at}\NormalTok{(}\FunctionTok{vars}\NormalTok{(lifeExp), }\FunctionTok{list}\NormalTok{(}\AttributeTok{Min =}\NormalTok{ min, }\AttributeTok{Med =}\NormalTok{ median, }\AttributeTok{Mean =}\NormalTok{ mean, }\AttributeTok{Max =}\NormalTok{ max, }\AttributeTok{Sd =}\NormalTok{ sd)) }\SpecialCharTok{\%\textgreater{}\%}
    \FunctionTok{data.frame}\NormalTok{()}
\NormalTok{)}
\end{Highlighting}
\end{Shaded}

\begin{longtable}[]{@{}lrrrrrr@{}}
\toprule\noalign{}
continent & year & Min & Med & Mean & Max & Sd \\
\midrule\noalign{}
\endhead
\bottomrule\noalign{}
\endlastfoot
Africa & 1952 & 30.000 & 38.8330 & 39.13550 & 52.724 & 5.1515814 \\
Africa & 1957 & 31.570 & 40.5925 & 41.26635 & 58.089 & 5.6201229 \\
Africa & 1962 & 32.767 & 42.6305 & 43.31944 & 60.246 & 5.8753639 \\
Africa & 1967 & 34.113 & 44.6985 & 45.33454 & 61.557 & 6.0826726 \\
Africa & 1972 & 35.400 & 47.0315 & 47.45094 & 64.274 & 6.4162583 \\
Africa & 1977 & 36.788 & 49.2725 & 49.58042 & 67.064 & 6.8081974 \\
Africa & 1982 & 38.445 & 50.7560 & 51.59287 & 69.885 & 7.3759401 \\
Africa & 1987 & 39.906 & 51.6395 & 53.34479 & 71.913 & 7.8640891 \\
Africa & 1992 & 23.599 & 52.4290 & 53.62958 & 73.615 & 9.4610710 \\
Africa & 1997 & 36.087 & 52.7590 & 53.59827 & 74.772 & 9.1033866 \\
Africa & 2002 & 39.193 & 51.2355 & 53.32523 & 75.744 & 9.5864959 \\
Africa & 2007 & 39.613 & 52.9265 & 54.80604 & 76.442 & 9.6307807 \\
Americas & 1952 & 37.579 & 54.7450 & 53.27984 & 68.750 & 9.3260819 \\
Americas & 1957 & 40.696 & 56.0740 & 55.96028 & 69.960 & 9.0331923 \\
Americas & 1962 & 43.428 & 58.2990 & 58.39876 & 71.300 & 8.5035437 \\
Americas & 1967 & 45.032 & 60.5230 & 60.41092 & 72.130 & 7.9091710 \\
Americas & 1972 & 46.714 & 63.4410 & 62.39492 & 72.880 & 7.3230168 \\
Americas & 1977 & 49.923 & 66.3530 & 64.39156 & 74.210 & 7.0694956 \\
Americas & 1982 & 51.461 & 67.4050 & 66.22884 & 75.760 & 6.7208338 \\
Americas & 1987 & 53.636 & 69.4980 & 68.09072 & 76.860 & 5.8019288 \\
Americas & 1992 & 55.089 & 69.8620 & 69.56836 & 77.950 & 5.1671038 \\
Americas & 1997 & 56.671 & 72.1460 & 71.15048 & 78.610 & 4.8875839 \\
Americas & 2002 & 58.137 & 72.0470 & 72.42204 & 79.770 & 4.7997055 \\
Americas & 2007 & 60.916 & 72.8990 & 73.60812 & 80.653 & 4.4409476 \\
Asia & 1952 & 28.801 & 44.8690 & 46.31439 & 65.390 & 9.2917507 \\
Asia & 1957 & 30.332 & 48.2840 & 49.31854 & 67.840 & 9.6354286 \\
Asia & 1962 & 31.997 & 49.3250 & 51.56322 & 69.390 & 9.8206319 \\
Asia & 1967 & 34.020 & 53.6550 & 54.66364 & 71.430 & 9.6509646 \\
Asia & 1972 & 36.088 & 56.9500 & 57.31927 & 73.420 & 9.7227000 \\
Asia & 1977 & 31.220 & 60.7650 & 59.61056 & 75.380 & 10.0221970 \\
Asia & 1982 & 39.854 & 63.7390 & 62.61794 & 77.110 & 8.5352214 \\
Asia & 1987 & 40.822 & 66.2950 & 64.85118 & 78.670 & 8.2037919 \\
Asia & 1992 & 41.674 & 68.6900 & 66.53721 & 79.360 & 8.0755490 \\
Asia & 1997 & 41.763 & 70.2650 & 68.02052 & 80.690 & 8.0911706 \\
Asia & 2002 & 42.129 & 71.0280 & 69.23388 & 82.000 & 8.3745954 \\
Asia & 2007 & 43.828 & 72.3960 & 70.72848 & 82.603 & 7.9637245 \\
Europe & 1952 & 43.585 & 65.9000 & 64.40850 & 72.670 & 6.3610883 \\
Europe & 1957 & 48.079 & 67.6500 & 66.70307 & 73.470 & 5.2958054 \\
Europe & 1962 & 52.098 & 69.5250 & 68.53923 & 73.680 & 4.3024996 \\
Europe & 1967 & 54.336 & 70.6100 & 69.73760 & 74.160 & 3.7997285 \\
Europe & 1972 & 57.005 & 70.8850 & 70.77503 & 74.720 & 3.2405764 \\
Europe & 1977 & 59.507 & 72.3350 & 71.93777 & 76.110 & 3.1210300 \\
Europe & 1982 & 61.036 & 73.4900 & 72.80640 & 76.990 & 3.2182603 \\
Europe & 1987 & 63.108 & 74.8150 & 73.64217 & 77.410 & 3.1696803 \\
Europe & 1992 & 66.146 & 75.4510 & 74.44010 & 78.770 & 3.2097811 \\
Europe & 1997 & 68.835 & 76.1160 & 75.50517 & 79.390 & 3.1046766 \\
Europe & 2002 & 70.845 & 77.5365 & 76.70060 & 80.620 & 2.9221796 \\
Europe & 2007 & 71.777 & 78.6085 & 77.64860 & 81.757 & 2.9798127 \\
Oceania & 1952 & 69.120 & 69.2550 & 69.25500 & 69.390 & 0.1909188 \\
Oceania & 1957 & 70.260 & 70.2950 & 70.29500 & 70.330 & 0.0494975 \\
Oceania & 1962 & 70.930 & 71.0850 & 71.08500 & 71.240 & 0.2192031 \\
Oceania & 1967 & 71.100 & 71.3100 & 71.31000 & 71.520 & 0.2969848 \\
Oceania & 1972 & 71.890 & 71.9100 & 71.91000 & 71.930 & 0.0282843 \\
Oceania & 1977 & 72.220 & 72.8550 & 72.85500 & 73.490 & 0.8980256 \\
Oceania & 1982 & 73.840 & 74.2900 & 74.29000 & 74.740 & 0.6363961 \\
Oceania & 1987 & 74.320 & 75.3200 & 75.32000 & 76.320 & 1.4142136 \\
Oceania & 1992 & 76.330 & 76.9450 & 76.94500 & 77.560 & 0.8697413 \\
Oceania & 1997 & 77.550 & 78.1900 & 78.19000 & 78.830 & 0.9050967 \\
Oceania & 2002 & 79.110 & 79.7400 & 79.74000 & 80.370 & 0.8909545 \\
Oceania & 2007 & 80.204 & 80.7195 & 80.71950 & 81.235 & 0.7290271 \\
\end{longtable}

\subsection{Task d}\label{task-d}

\begin{Shaded}
\begin{Highlighting}[]
\NormalTok{gapminder }\SpecialCharTok{\%\textgreater{}\%}
  \FunctionTok{group\_by}\NormalTok{(continent, year) }\SpecialCharTok{\%\textgreater{}\%}
  \FunctionTok{summarise}\NormalTok{(}\AttributeTok{Mean\_lifeExp=}\FunctionTok{mean}\NormalTok{(lifeExp, }\AttributeTok{na.rm =}\NormalTok{ T), }\AttributeTok{SD\_lifeExp=}\FunctionTok{sd}\NormalTok{(lifeExp, }\AttributeTok{na.rm =}\NormalTok{ T), }\AttributeTok{.groups =} \StringTok{"drop"}\NormalTok{) }\SpecialCharTok{\%\textgreater{}\%}
  \FunctionTok{ggplot}\NormalTok{(}\FunctionTok{aes}\NormalTok{(}\AttributeTok{x=}\NormalTok{year, }\AttributeTok{y=}\NormalTok{Mean\_lifeExp)) }\SpecialCharTok{+}
  \FunctionTok{geom\_ribbon}\NormalTok{(}\FunctionTok{aes}\NormalTok{(}\AttributeTok{ymin=}\NormalTok{ Mean\_lifeExp }\SpecialCharTok{{-}}\NormalTok{ SD\_lifeExp, }\AttributeTok{ymax =}\NormalTok{ Mean\_lifeExp }\SpecialCharTok{+}\NormalTok{ SD\_lifeExp), }\AttributeTok{fill =} \StringTok{"grey70"}\NormalTok{) }\SpecialCharTok{+}
  \FunctionTok{geom\_line}\NormalTok{() }\SpecialCharTok{+}
  \FunctionTok{facet\_grid}\NormalTok{(.}\SpecialCharTok{\textasciitilde{}}\NormalTok{continent) }\SpecialCharTok{+}
  \FunctionTok{theme}\NormalTok{(}\AttributeTok{axis.text.x =} \FunctionTok{element\_text}\NormalTok{(}\AttributeTok{angle=}\DecValTok{90}\NormalTok{)) }\SpecialCharTok{+}
  \FunctionTok{ylim}\NormalTok{(}\DecValTok{0}\NormalTok{, }\ConstantTok{NA}\NormalTok{)}
\end{Highlighting}
\end{Shaded}

\includegraphics{Ass1_files/figure-latex/E1_d-1.pdf}

\section{Exercise 2 - Elementary data analysis and model
training}\label{exercise-2---elementary-data-analysis-and-model-training}

\subsection{Task a}\label{task-a-1}

\begin{Shaded}
\begin{Highlighting}[]
\NormalTok{weatherHistory }\OtherTok{\textless{}{-}} \FunctionTok{read.csv}\NormalTok{(}\StringTok{"weatherHistory.csv"}\NormalTok{)}
\FunctionTok{head}\NormalTok{(weatherHistory)}
\end{Highlighting}
\end{Shaded}

\begin{verbatim}
##                  Formatted.Date       Summary Precip.Type Temperature..C.
## 1 2006-04-01 00:00:00.000 +0200 Partly Cloudy        rain        9.472222
## 2 2006-04-01 01:00:00.000 +0200 Partly Cloudy        rain        9.355556
## 3 2006-04-01 02:00:00.000 +0200 Mostly Cloudy        rain        9.377778
## 4 2006-04-01 03:00:00.000 +0200 Partly Cloudy        rain        8.288889
## 5 2006-04-01 04:00:00.000 +0200 Mostly Cloudy        rain        8.755556
## 6 2006-04-01 05:00:00.000 +0200 Partly Cloudy        rain        9.222222
##   Apparent.Temperature..C. Humidity Wind.Speed..km.h. Wind.Bearing..degrees.
## 1                 7.388889     0.89           14.1197                    251
## 2                 7.227778     0.86           14.2646                    259
## 3                 9.377778     0.89            3.9284                    204
## 4                 5.944444     0.83           14.1036                    269
## 5                 6.977778     0.83           11.0446                    259
## 6                 7.111111     0.85           13.9587                    258
##   Visibility..km. Loud.Cover Pressure..millibars.
## 1         15.8263          0              1015.13
## 2         15.8263          0              1015.63
## 3         14.9569          0              1015.94
## 4         15.8263          0              1016.41
## 5         15.8263          0              1016.51
## 6         14.9569          0              1016.66
##                       Daily.Summary
## 1 Partly cloudy throughout the day.
## 2 Partly cloudy throughout the day.
## 3 Partly cloudy throughout the day.
## 4 Partly cloudy throughout the day.
## 5 Partly cloudy throughout the day.
## 6 Partly cloudy throughout the day.
\end{verbatim}

\textbf{Qualitative nominal}

\begin{itemize}
\tightlist
\item
  Summary
\item
  Precip.Type
\item
  Daily.Summary
\end{itemize}

\textbf{Quantitative Continuous:}

\begin{itemize}
\tightlist
\item
  Temperature..C.
\item
  Apparent.Temperature..C.
\item
  Humidity
\item
  Wind.Speed..km.h.
\item
  Visibility..km.
\item
  Wind.Bearing..degrees (Reason: Not ranked)
\end{itemize}

\textbf{Quantitative Discrete:}

\begin{itemize}
\tightlist
\item
  Formatted.Date
\item
  Loud.Cover
\end{itemize}

\subsubsection{Qualitative nominal}\label{qualitative-nominal}

\begin{Shaded}
\begin{Highlighting}[]
\NormalTok{weatherHistory }\SpecialCharTok{\%\textgreater{}\%}
  \FunctionTok{group\_by}\NormalTok{(Precip.Type) }\SpecialCharTok{\%\textgreater{}\%}
  \FunctionTok{summarize}\NormalTok{(}\AttributeTok{Count =} \FunctionTok{n}\NormalTok{()) }\SpecialCharTok{\%\textgreater{}\%}
  \FunctionTok{ggplot}\NormalTok{(}\FunctionTok{aes}\NormalTok{(}\AttributeTok{x=}\NormalTok{Precip.Type, }\AttributeTok{y=}\NormalTok{Count)) }\SpecialCharTok{+}
  \FunctionTok{geom\_bar}\NormalTok{(}\AttributeTok{stat=}\StringTok{\textquotesingle{}identity\textquotesingle{}}\NormalTok{, }\AttributeTok{position=}\StringTok{\textquotesingle{}dodge\textquotesingle{}}\NormalTok{)}
\end{Highlighting}
\end{Shaded}

\includegraphics{Ass1_files/figure-latex/E2_a_2-1.pdf}

\begin{Shaded}
\begin{Highlighting}[]
\NormalTok{weatherHistory }\SpecialCharTok{\%\textgreater{}\%}
  \FunctionTok{group\_by}\NormalTok{(Summary) }\SpecialCharTok{\%\textgreater{}\%}
  \FunctionTok{summarize}\NormalTok{(}\AttributeTok{Count =} \FunctionTok{n}\NormalTok{()) }\SpecialCharTok{\%\textgreater{}\%}
  \FunctionTok{ggplot}\NormalTok{(}\FunctionTok{aes}\NormalTok{(}\AttributeTok{x=}\NormalTok{Summary, }\AttributeTok{y=}\NormalTok{Count)) }\SpecialCharTok{+}
  \FunctionTok{geom\_bar}\NormalTok{(}\AttributeTok{stat=}\StringTok{\textquotesingle{}identity\textquotesingle{}}\NormalTok{, }\AttributeTok{position=}\StringTok{\textquotesingle{}dodge\textquotesingle{}}\NormalTok{) }\SpecialCharTok{+}
  \FunctionTok{theme}\NormalTok{(}\AttributeTok{axis.text.x =} \FunctionTok{element\_text}\NormalTok{(}\AttributeTok{angle=}\DecValTok{90}\NormalTok{))}
\end{Highlighting}
\end{Shaded}

\includegraphics{Ass1_files/figure-latex/E2_a_3-1.pdf}

\begin{Shaded}
\begin{Highlighting}[]
\NormalTok{weatherHistory }\SpecialCharTok{\%\textgreater{}\%}
  \FunctionTok{group\_by}\NormalTok{(Daily.Summary) }\SpecialCharTok{\%\textgreater{}\%}
  \FunctionTok{summarize}\NormalTok{(}\AttributeTok{Count =} \FunctionTok{n}\NormalTok{()) }\SpecialCharTok{\%\textgreater{}\%}
  \FunctionTok{ggplot}\NormalTok{(}\FunctionTok{aes}\NormalTok{(}\AttributeTok{x=}\NormalTok{Daily.Summary, }\AttributeTok{y=}\NormalTok{Count)) }\SpecialCharTok{+}
  \FunctionTok{geom\_bar}\NormalTok{(}\AttributeTok{stat=}\StringTok{\textquotesingle{}identity\textquotesingle{}}\NormalTok{, }\AttributeTok{position=}\StringTok{\textquotesingle{}dodge\textquotesingle{}}\NormalTok{)}
\end{Highlighting}
\end{Shaded}

\includegraphics{Ass1_files/figure-latex/E2_a_4-1.pdf}

\subsubsection{Discrete nominal}\label{discrete-nominal}

\begin{Shaded}
\begin{Highlighting}[]
\NormalTok{weatherHistory }\SpecialCharTok{\%\textgreater{}\%}
  \FunctionTok{ggplot}\NormalTok{(}\FunctionTok{aes}\NormalTok{(Temperature..C.)) }\SpecialCharTok{+}
  \FunctionTok{geom\_histogram}\NormalTok{(}\FunctionTok{aes}\NormalTok{(}\AttributeTok{y =}\NormalTok{ ..density..), }\AttributeTok{fill =} \StringTok{"white"}\NormalTok{, }\AttributeTok{color=}\StringTok{"black"}\NormalTok{) }\SpecialCharTok{+}
  \FunctionTok{stat\_density}\NormalTok{(}\AttributeTok{kernel =} \StringTok{"gaussian"}\NormalTok{, }\AttributeTok{fill =} \ConstantTok{NA}\NormalTok{, }\AttributeTok{colour =} \StringTok{"black"}\NormalTok{)}
\end{Highlighting}
\end{Shaded}

\begin{verbatim}
## Warning: The dot-dot notation (`..density..`) was deprecated in ggplot2 3.4.0.
## i Please use `after_stat(density)` instead.
## This warning is displayed once every 8 hours.
## Call `lifecycle::last_lifecycle_warnings()` to see where this warning was
## generated.
\end{verbatim}

\begin{verbatim}
## `stat_bin()` using `bins = 30`. Pick better value with `binwidth`.
\end{verbatim}

\includegraphics{Ass1_files/figure-latex/E2_a_5-1.pdf}

\begin{Shaded}
\begin{Highlighting}[]
\NormalTok{weatherHistory }\SpecialCharTok{\%\textgreater{}\%}
  \FunctionTok{ggplot}\NormalTok{(}\FunctionTok{aes}\NormalTok{(Apparent.Temperature..C.)) }\SpecialCharTok{+}
  \FunctionTok{geom\_histogram}\NormalTok{(}\FunctionTok{aes}\NormalTok{(}\AttributeTok{y =}\NormalTok{ ..density..), }\AttributeTok{fill =} \StringTok{"white"}\NormalTok{, }\AttributeTok{color=}\StringTok{"black"}\NormalTok{) }\SpecialCharTok{+}
  \FunctionTok{stat\_density}\NormalTok{(}\AttributeTok{kernel =} \StringTok{"gaussian"}\NormalTok{, }\AttributeTok{fill =} \ConstantTok{NA}\NormalTok{, }\AttributeTok{colour =} \StringTok{"black"}\NormalTok{)}
\end{Highlighting}
\end{Shaded}

\begin{verbatim}
## `stat_bin()` using `bins = 30`. Pick better value with `binwidth`.
\end{verbatim}

\includegraphics{Ass1_files/figure-latex/E2_a_6-1.pdf}

\begin{Shaded}
\begin{Highlighting}[]
\NormalTok{weatherHistory }\SpecialCharTok{\%\textgreater{}\%}
  \FunctionTok{ggplot}\NormalTok{(}\FunctionTok{aes}\NormalTok{(Humidity)) }\SpecialCharTok{+}
  \FunctionTok{geom\_histogram}\NormalTok{(}\FunctionTok{aes}\NormalTok{(}\AttributeTok{y =}\NormalTok{ ..density..), }\AttributeTok{fill =} \StringTok{"white"}\NormalTok{, }\AttributeTok{color=}\StringTok{"black"}\NormalTok{) }\SpecialCharTok{+}
  \FunctionTok{stat\_density}\NormalTok{(}\AttributeTok{kernel =} \StringTok{"gaussian"}\NormalTok{, }\AttributeTok{fill =} \ConstantTok{NA}\NormalTok{, }\AttributeTok{colour =} \StringTok{"black"}\NormalTok{)}
\end{Highlighting}
\end{Shaded}

\begin{verbatim}
## `stat_bin()` using `bins = 30`. Pick better value with `binwidth`.
\end{verbatim}

\includegraphics{Ass1_files/figure-latex/E2_a_7-1.pdf}

\begin{Shaded}
\begin{Highlighting}[]
\NormalTok{weatherHistory }\SpecialCharTok{\%\textgreater{}\%}
  \FunctionTok{ggplot}\NormalTok{(}\FunctionTok{aes}\NormalTok{(Wind.Speed..km.h.)) }\SpecialCharTok{+}
  \FunctionTok{geom\_histogram}\NormalTok{(}\FunctionTok{aes}\NormalTok{(}\AttributeTok{y =}\NormalTok{ ..density..), }\AttributeTok{fill =} \StringTok{"white"}\NormalTok{, }\AttributeTok{color=}\StringTok{"black"}\NormalTok{) }\SpecialCharTok{+}
  \FunctionTok{stat\_density}\NormalTok{(}\AttributeTok{kernel =} \StringTok{"gaussian"}\NormalTok{, }\AttributeTok{fill =} \ConstantTok{NA}\NormalTok{, }\AttributeTok{colour =} \StringTok{"black"}\NormalTok{)}
\end{Highlighting}
\end{Shaded}

\begin{verbatim}
## `stat_bin()` using `bins = 30`. Pick better value with `binwidth`.
\end{verbatim}

\includegraphics{Ass1_files/figure-latex/E2_a_8-1.pdf}

\begin{Shaded}
\begin{Highlighting}[]
\NormalTok{weatherHistory }\SpecialCharTok{\%\textgreater{}\%}
  \FunctionTok{ggplot}\NormalTok{(}\FunctionTok{aes}\NormalTok{(Visibility..km.)) }\SpecialCharTok{+}
  \FunctionTok{geom\_histogram}\NormalTok{(}\FunctionTok{aes}\NormalTok{(}\AttributeTok{y =}\NormalTok{ ..density..), }\AttributeTok{fill =} \StringTok{"white"}\NormalTok{, }\AttributeTok{color=}\StringTok{"black"}\NormalTok{) }\SpecialCharTok{+}
  \FunctionTok{stat\_density}\NormalTok{(}\AttributeTok{kernel =} \StringTok{"gaussian"}\NormalTok{, }\AttributeTok{fill =} \ConstantTok{NA}\NormalTok{, }\AttributeTok{colour =} \StringTok{"black"}\NormalTok{)}
\end{Highlighting}
\end{Shaded}

\begin{verbatim}
## `stat_bin()` using `bins = 30`. Pick better value with `binwidth`.
\end{verbatim}

\includegraphics{Ass1_files/figure-latex/E2_a_9-1.pdf}

\begin{Shaded}
\begin{Highlighting}[]
\NormalTok{weatherHistory }\SpecialCharTok{\%\textgreater{}\%}
  \FunctionTok{ggplot}\NormalTok{(}\FunctionTok{aes}\NormalTok{(Wind.Bearing..degrees.)) }\SpecialCharTok{+}
  \FunctionTok{geom\_histogram}\NormalTok{(}\FunctionTok{aes}\NormalTok{(}\AttributeTok{y =}\NormalTok{ ..density..), }\AttributeTok{fill =} \StringTok{"white"}\NormalTok{, }\AttributeTok{color=}\StringTok{"black"}\NormalTok{) }\SpecialCharTok{+}
  \FunctionTok{stat\_density}\NormalTok{(}\AttributeTok{kernel =} \StringTok{"gaussian"}\NormalTok{, }\AttributeTok{fill =} \ConstantTok{NA}\NormalTok{, }\AttributeTok{colour =} \StringTok{"black"}\NormalTok{)}
\end{Highlighting}
\end{Shaded}

\begin{verbatim}
## `stat_bin()` using `bins = 30`. Pick better value with `binwidth`.
\end{verbatim}

\includegraphics{Ass1_files/figure-latex/E2_a_11-1.pdf}

\subsubsection{Quantative discrete}\label{quantative-discrete}

\begin{Shaded}
\begin{Highlighting}[]
\NormalTok{weatherHistory }\SpecialCharTok{\%\textgreater{}\%}
  \FunctionTok{group\_by}\NormalTok{(Loud.Cover) }\SpecialCharTok{\%\textgreater{}\%}
  \FunctionTok{summarize}\NormalTok{(}\AttributeTok{Count =} \FunctionTok{n}\NormalTok{()) }\SpecialCharTok{\%\textgreater{}\%}
  \FunctionTok{ggplot}\NormalTok{(}\FunctionTok{aes}\NormalTok{(}\AttributeTok{x=}\NormalTok{Loud.Cover, }\AttributeTok{y=}\NormalTok{Count)) }\SpecialCharTok{+}
  \FunctionTok{geom\_bar}\NormalTok{(}\AttributeTok{stat=}\StringTok{\textquotesingle{}identity\textquotesingle{}}\NormalTok{, }\AttributeTok{position=}\StringTok{\textquotesingle{}dodge\textquotesingle{}}\NormalTok{) }\SpecialCharTok{+}
  \FunctionTok{theme}\NormalTok{(}\AttributeTok{axis.text.x =} \FunctionTok{element\_text}\NormalTok{(}\AttributeTok{angle=}\DecValTok{90}\NormalTok{))}
\end{Highlighting}
\end{Shaded}

\includegraphics{Ass1_files/figure-latex/E2_a_10-1.pdf}

\subsection{Task b}\label{task-b-1}

First removing all columns that seem irrelevant, reasoning:

\begin{itemize}
\tightlist
\item
  Formatted.Date : When encoded it will be equal to row label (1, 2, 3,
  \ldots) which tells nothing
\item
  Loud.Cover : All values are 0, therfore tells nothing
\item
  Daily.Summary : Too big to onehotencode effectivly
\end{itemize}

Then remove all rows with NA, do this after removing irrelevant columns
so data is not lost to having NA in the removed columns

\begin{Shaded}
\begin{Highlighting}[]
\NormalTok{drop\_weatherHistory }\OtherTok{\textless{}{-}}\NormalTok{ weatherHistory }\SpecialCharTok{\%\textgreater{}\%}\NormalTok{ dplyr}\SpecialCharTok{::}\FunctionTok{select}\NormalTok{(}\SpecialCharTok{{-}}\FunctionTok{c}\NormalTok{(}\StringTok{"Formatted.Date"}\NormalTok{, }\StringTok{"Daily.Summary"}\NormalTok{, }\StringTok{"Loud.Cover"}\NormalTok{))}
\NormalTok{drop\_weatherHistory }\OtherTok{\textless{}{-}} \FunctionTok{na.omit}\NormalTok{(drop\_weatherHistory) }\CommentTok{\# Remove all NA}
\FunctionTok{head}\NormalTok{(drop\_weatherHistory)}
\end{Highlighting}
\end{Shaded}

\begin{verbatim}
##         Summary Precip.Type Temperature..C. Apparent.Temperature..C. Humidity
## 1 Partly Cloudy        rain        9.472222                 7.388889     0.89
## 2 Partly Cloudy        rain        9.355556                 7.227778     0.86
## 3 Mostly Cloudy        rain        9.377778                 9.377778     0.89
## 4 Partly Cloudy        rain        8.288889                 5.944444     0.83
## 5 Mostly Cloudy        rain        8.755556                 6.977778     0.83
## 6 Partly Cloudy        rain        9.222222                 7.111111     0.85
##   Wind.Speed..km.h. Wind.Bearing..degrees. Visibility..km. Pressure..millibars.
## 1           14.1197                    251         15.8263              1015.13
## 2           14.2646                    259         15.8263              1015.63
## 3            3.9284                    204         14.9569              1015.94
## 4           14.1036                    269         15.8263              1016.41
## 5           11.0446                    259         15.8263              1016.51
## 6           13.9587                    258         14.9569              1016.66
\end{verbatim}

\begin{Shaded}
\begin{Highlighting}[]
\NormalTok{num\_wH }\OtherTok{\textless{}{-}}\NormalTok{ drop\_weatherHistory }\SpecialCharTok{\%\textgreater{}\%}
\NormalTok{  dplyr}\SpecialCharTok{::}\FunctionTok{select}\NormalTok{(}\SpecialCharTok{{-}}\FunctionTok{c}\NormalTok{(}\StringTok{"Summary"}\NormalTok{, }\StringTok{"Precip.Type"}\NormalTok{)) }
\NormalTok{num\_stand\_wH }\OtherTok{\textless{}{-}} \FunctionTok{as.data.frame}\NormalTok{(}\FunctionTok{sapply}\NormalTok{(num\_wH, }\ControlFlowTok{function}\NormalTok{(x) ((x}\SpecialCharTok{{-}}\FunctionTok{mean}\NormalTok{(x))}\SpecialCharTok{/}\FunctionTok{sd}\NormalTok{(x))))}

\NormalTok{qualitative\_wH }\OtherTok{\textless{}{-}}\NormalTok{ drop\_weatherHistory }\SpecialCharTok{\%\textgreater{}\%}
\NormalTok{  dplyr}\SpecialCharTok{::}\FunctionTok{select}\NormalTok{(}\FunctionTok{c}\NormalTok{(}\StringTok{"Summary"}\NormalTok{, }\StringTok{"Precip.Type"}\NormalTok{)) }\CommentTok{\#Omitted "Formatted.Date","Daily.Summary"}

\NormalTok{q1 }\OtherTok{\textless{}{-}} \FunctionTok{table}\NormalTok{(}\DecValTok{1}\SpecialCharTok{:}\FunctionTok{nrow}\NormalTok{(drop\_weatherHistory), drop\_weatherHistory}\SpecialCharTok{$}\NormalTok{Precip.Type) }\CommentTok{\# as.data.frame.matrix(}
\NormalTok{q2 }\OtherTok{\textless{}{-}} \FunctionTok{table}\NormalTok{(}\DecValTok{1}\SpecialCharTok{:}\FunctionTok{nrow}\NormalTok{(drop\_weatherHistory), drop\_weatherHistory}\SpecialCharTok{$}\NormalTok{Summary)}
\NormalTok{q }\OtherTok{\textless{}{-}} \FunctionTok{as.data.frame.matrix}\NormalTok{(}\FunctionTok{cbind}\NormalTok{(q1, q2))}

\NormalTok{cleaned\_wH }\OtherTok{\textless{}{-}} \FunctionTok{cbind}\NormalTok{(num\_stand\_wH, q)}
\FunctionTok{head}\NormalTok{(cleaned\_wH)}
\end{Highlighting}
\end{Shaded}

\begin{verbatim}
##   Temperature..C. Apparent.Temperature..C.  Humidity Wind.Speed..km.h.
## 1      -0.2575977               -0.3240338 0.7934663        0.47863251
## 2      -0.2698121               -0.3390953 0.6399922        0.49959129
## 3      -0.2674856               -0.1381015 0.7934663       -0.99546821
## 4      -0.3814869               -0.4590684 0.4865181        0.47630376
## 5      -0.3326292               -0.3624667 0.4865181        0.03384067
## 6      -0.2837715               -0.3500020 0.5888342        0.45534498
##   Wind.Bearing..degrees. Visibility..km. Pressure..millibars. null rain snow
## 1              0.5912529        1.306969            0.1016847    0    1    0
## 2              0.6657523        1.306969            0.1059593    0    1    0
## 3              0.1535690        1.099580            0.1086095    0    1    0
## 4              0.7588766        1.306969            0.1126276    0    1    0
## 5              0.6657523        1.306969            0.1134826    0    1    0
## 6              0.6564399        1.099580            0.1147649    0    1    0
##   Breezy Breezy and Dry Breezy and Foggy Breezy and Mostly Cloudy
## 1      0              0                0                        0
## 2      0              0                0                        0
## 3      0              0                0                        0
## 4      0              0                0                        0
## 5      0              0                0                        0
## 6      0              0                0                        0
##   Breezy and Overcast Breezy and Partly Cloudy Clear
## 1                   0                        0     0
## 2                   0                        0     0
## 3                   0                        0     0
## 4                   0                        0     0
## 5                   0                        0     0
## 6                   0                        0     0
##   Dangerously Windy and Partly Cloudy Drizzle Dry Dry and Mostly Cloudy
## 1                                   0       0   0                     0
## 2                                   0       0   0                     0
## 3                                   0       0   0                     0
## 4                                   0       0   0                     0
## 5                                   0       0   0                     0
## 6                                   0       0   0                     0
##   Dry and Partly Cloudy Foggy Humid and Mostly Cloudy Humid and Overcast
## 1                     0     0                       0                  0
## 2                     0     0                       0                  0
## 3                     0     0                       0                  0
## 4                     0     0                       0                  0
## 5                     0     0                       0                  0
## 6                     0     0                       0                  0
##   Humid and Partly Cloudy Light Rain Mostly Cloudy Overcast Partly Cloudy Rain
## 1                       0          0             0        0             1    0
## 2                       0          0             0        0             1    0
## 3                       0          0             1        0             0    0
## 4                       0          0             0        0             1    0
## 5                       0          0             1        0             0    0
## 6                       0          0             0        0             1    0
##   Windy Windy and Dry Windy and Foggy Windy and Mostly Cloudy
## 1     0             0               0                       0
## 2     0             0               0                       0
## 3     0             0               0                       0
## 4     0             0               0                       0
## 5     0             0               0                       0
## 6     0             0               0                       0
##   Windy and Overcast Windy and Partly Cloudy
## 1                  0                       0
## 2                  0                       0
## 3                  0                       0
## 4                  0                       0
## 5                  0                       0
## 6                  0                       0
\end{verbatim}

\begin{Shaded}
\begin{Highlighting}[]
\NormalTok{sample }\OtherTok{\textless{}{-}} \FunctionTok{sample}\NormalTok{(}\FunctionTok{c}\NormalTok{(T, F), }\FunctionTok{nrow}\NormalTok{(cleaned\_wH), }\AttributeTok{replace=}\NormalTok{T, }\AttributeTok{prob=}\FunctionTok{c}\NormalTok{(}\FloatTok{0.75}\NormalTok{, }\FloatTok{0.25}\NormalTok{))}
\NormalTok{test\_wH }\OtherTok{\textless{}{-}}\NormalTok{ cleaned\_wH[}\SpecialCharTok{!}\NormalTok{sample,]}
\NormalTok{train\_wH }\OtherTok{\textless{}{-}}\NormalTok{ cleaned\_wH[sample,]}
\end{Highlighting}
\end{Shaded}

\subsection{Task c}\label{task-c-1}

Reason for chosen variables:

\begin{itemize}
\tightlist
\item
  Temperate (C) : Baseline that gets moved
\item
  Humidity : Feels a lot hotter when its more humid, harder to sweat
\item
  Wind speed : Wind makes skin feel colder
\item
  Pressure : Pressure changes based on if it may rain or not, feels
  different
\item
  Rain/Snow : If it rains the air feels colder
\end{itemize}

\begin{Shaded}
\begin{Highlighting}[]
\NormalTok{wH\_lm }\OtherTok{\textless{}{-}}\NormalTok{ train\_wH }\SpecialCharTok{\%\textgreater{}\%}
  \FunctionTok{lm}\NormalTok{(Apparent.Temperature..C. }\SpecialCharTok{\textasciitilde{}}\NormalTok{ rain }\SpecialCharTok{+}\NormalTok{ snow }\SpecialCharTok{+}\NormalTok{ Pressure..millibars. }\SpecialCharTok{+}\NormalTok{ Humidity }\SpecialCharTok{+}\NormalTok{ Temperature..C. }\SpecialCharTok{+}\NormalTok{ Wind.Speed..km.h., .)}

\FunctionTok{summary.aov}\NormalTok{(wH\_lm)}
\end{Highlighting}
\end{Shaded}

\begin{verbatim}
##                         Df Sum Sq Mean Sq F value Pr(>F)    
## rain                     1  22548   22548 2238282 <2e-16 ***
## snow                     1    785     785   77921 <2e-16 ***
## Pressure..millibars.     1      2       2     214 <2e-16 ***
## Humidity                 1  17058   17058 1693291 <2e-16 ***
## Temperature..C.          1  31169   31169 3094079 <2e-16 ***
## Wind.Speed..km.h.        1    255     255   25294 <2e-16 ***
## Residuals            72385    729       0                   
## ---
## Signif. codes:  0 '***' 0.001 '**' 0.01 '*' 0.05 '.' 0.1 ' ' 1
\end{verbatim}

\begin{Shaded}
\begin{Highlighting}[]
\FunctionTok{summary}\NormalTok{(wH\_lm)}
\end{Highlighting}
\end{Shaded}

\begin{verbatim}
## 
## Call:
## lm(formula = Apparent.Temperature..C. ~ rain + snow + Pressure..millibars. + 
##     Humidity + Temperature..C. + Wind.Speed..km.h., data = .)
## 
## Residuals:
##      Min       1Q   Median       3Q      Max 
## -0.39581 -0.06833 -0.00954  0.06153  0.50383 
## 
## Coefficients:
##                        Estimate Std. Error  t value Pr(>|t|)    
## (Intercept)           0.0412733  0.0051298    8.046 8.69e-16 ***
## rain                 -0.0370676  0.0051474   -7.201 6.02e-13 ***
## snow                 -0.0776405  0.0052985  -14.653  < 2e-16 ***
## Pressure..millibars.  0.0020225  0.0003750    5.393 6.96e-08 ***
## Humidity              0.0161676  0.0005163   31.315  < 2e-16 ***
## Temperature..C.       0.9961083  0.0005934 1678.596  < 2e-16 ***
## Wind.Speed..km.h.    -0.0626910  0.0003942 -159.042  < 2e-16 ***
## ---
## Signif. codes:  0 '***' 0.001 '**' 0.01 '*' 0.05 '.' 0.1 ' ' 1
## 
## Residual standard error: 0.1004 on 72385 degrees of freedom
## Multiple R-squared:  0.9899, Adjusted R-squared:  0.9899 
## F-statistic: 1.188e+06 on 6 and 72385 DF,  p-value: < 2.2e-16
\end{verbatim}

As can be seen an the ANOVA and t test for the different values, they
are all significant within \(\alpha \approx 0\) which means that there
is almost 0 chance that the factors are due to random chance. (FIX
LATER, DOUBBLE CHECK)

\begin{Shaded}
\begin{Highlighting}[]
\NormalTok{y\_test\_true }\OtherTok{\textless{}{-}}\NormalTok{ test\_wH}\SpecialCharTok{$}\NormalTok{Apparent.Temperature..C.}
\NormalTok{y\_test\_pred }\OtherTok{\textless{}{-}} \FunctionTok{predict}\NormalTok{(wH\_lm, }\AttributeTok{newdata =}\NormalTok{ test\_wH)}
\NormalTok{y\_train\_true }\OtherTok{\textless{}{-}}\NormalTok{ train\_wH}\SpecialCharTok{$}\NormalTok{Apparent.Temperature..C.}
\NormalTok{y\_train\_pred }\OtherTok{\textless{}{-}} \FunctionTok{predict}\NormalTok{(wH\_lm, }\AttributeTok{newdata =}\NormalTok{ train\_wH)}
\end{Highlighting}
\end{Shaded}

\subsubsection{RMSE}\label{rmse}

\[ \text{RMSE}(y, \hat{y}) = \sqrt{\frac{\sum_{i=0}^{N - 1} (y_i - \hat{y}_i)^2}{N}} \]

\begin{Shaded}
\begin{Highlighting}[]
\NormalTok{rmse\_test }\OtherTok{\textless{}{-}} \FunctionTok{RMSE}\NormalTok{(y\_test\_pred, y\_test\_true)}
\NormalTok{rmse\_train }\OtherTok{\textless{}{-}} \FunctionTok{RMSE}\NormalTok{(y\_train\_pred, y\_train\_true)}
\NormalTok{rmse\_test}
\end{Highlighting}
\end{Shaded}

\begin{verbatim}
## [1] 0.1003458
\end{verbatim}

\begin{Shaded}
\begin{Highlighting}[]
\NormalTok{rmse\_test}
\end{Highlighting}
\end{Shaded}

\begin{verbatim}
## [1] 0.1003458
\end{verbatim}

\subsubsection{MAE}\label{mae}

\[ \text{MAE}(x,y) = \sum_{i=1}^{D}|x_i-y_i|  \]

\begin{Shaded}
\begin{Highlighting}[]
\NormalTok{mae\_test }\OtherTok{\textless{}{-}} \FunctionTok{MAE}\NormalTok{(y\_test\_pred, y\_test\_true)}
\NormalTok{mae\_train }\OtherTok{\textless{}{-}} \FunctionTok{MAE}\NormalTok{(y\_train\_pred, y\_train\_true)}
\NormalTok{mae\_test}
\end{Highlighting}
\end{Shaded}

\begin{verbatim}
## [1] 0.07899319
\end{verbatim}

\begin{Shaded}
\begin{Highlighting}[]
\NormalTok{mae\_train}
\end{Highlighting}
\end{Shaded}

\begin{verbatim}
## [1] 0.078913
\end{verbatim}

\subsubsection{\texorpdfstring{\(\text{R}^{2}\) score (coefficient of
determination)}{\textbackslash text\{R\}\^{}\{2\} score (coefficient of determination)}}\label{textr2-score-coefficient-of-determination}

\[ 
R^2 = 1 - \frac{\text{SSR (sum of square regression)}}{\text{SST (total sum of squares)}} 
= 1 - \frac{\sum_{i=0}^{N - 1} (y_i - \hat{y}_i)^2}{\sum_{i=0}^{N - 1} (y_i - \hat{y})^2}
\]

\begin{Shaded}
\begin{Highlighting}[]
\NormalTok{R2\_test }\OtherTok{\textless{}{-}} \FunctionTok{R2}\NormalTok{(y\_test\_pred, y\_test\_true)}
\NormalTok{R2\_train }\OtherTok{\textless{}{-}} \FunctionTok{R2}\NormalTok{(y\_train\_pred, y\_train\_true)}
\NormalTok{R2\_test}
\end{Highlighting}
\end{Shaded}

\begin{verbatim}
## [1] 0.9898663
\end{verbatim}

\begin{Shaded}
\begin{Highlighting}[]
\NormalTok{R2\_train}
\end{Highlighting}
\end{Shaded}

\begin{verbatim}
## [1] 0.9899486
\end{verbatim}

\section{Exercise 3 - Linear Regression and Diagnostic
Plots}\label{exercise-3---linear-regression-and-diagnostic-plots}

\subsection{Task a}\label{task-a-2}

\subsubsection{a) Linearity}\label{a-linearity}

The relationship between X and Y is linear.

If a function is linear, then all factors increase at a consistent rate
for each step. Gradient and all partial first order derivatives will be
a function of constants Linear function: \(y(x,z) = 2x + 4z + 5xz + 10\)
Non-Linear function: \(y(x,z) = 2x^2 + 5z^5\)

\subsubsection{b) Homoscedasticity}\label{b-homoscedasticity}

Residuals have constant variance.

homogeneity of variance assumes that all observations are picked from a
sources that have equal variance. In other words the data points around
a linear model should vary equally from the line, if there are any cone
shape or other such irregularities this assumption is broken and the
model that will be produced will eb flawed.

\subsubsection{c) Independence}\label{c-independence}

Residuals are independent.

Observations can't depend on each other, in other words if you pick one
observation from a population for your sample, this should not affect
the next sample you choose. In other words no observations should depend
or affect each other.

By the very nature of lm, it is \textbf{assumed} that you have a i.i.d
dataset. This has to be done in the sampeling stage, not the cleaning
stage.

\subsubsection{d) Normaility}\label{d-normaility}

Residuals are normally distributed.

Assumes that the data follows a normal distribution.

Can be tested by making a Q-Q plot and seeing how well the data points
follows the line. If the right tail is very heavy, you probably should
log the value slightly heavy tails can still be used because of the law
of large numbers.

\subsection{Task b}\label{task-b-2}

These are all diagnostic plots that allow us to test our assumptions

\subsubsection{Residuals vs Fitted}\label{residuals-vs-fitted}

A scatter plot where residuals are on the y-axis and fitted values are
on the x-axis. Used to detect non-linearity, unequal error variances,
and outliers.

\begin{Shaded}
\begin{Highlighting}[]
\FunctionTok{plot}\NormalTok{(wH\_lm,}\DecValTok{1}\NormalTok{)}
\end{Highlighting}
\end{Shaded}

\includegraphics{Ass1_files/figure-latex/E3_b_1-1.pdf}

\subsubsection{Normal Q-Q}\label{normal-q-q}

Points on the Normal Q-Q plot provide an indication of univariate
normality of the dataset. If the data is normally distributed, the
points will fall on the line, otherwise it implies that the assumption
of Normality is broken. To test for normality of residuals

\begin{Shaded}
\begin{Highlighting}[]
\FunctionTok{plot}\NormalTok{(wH\_lm,}\DecValTok{2}\NormalTok{)}
\end{Highlighting}
\end{Shaded}

\includegraphics{Ass1_files/figure-latex/E3_b_2-1.pdf}

\subsubsection{Scale-Location (or
Spread-Location)}\label{scale-location-or-spread-location}

Simmilar to residuals vs fitted, but instead of using residuals on the
y-axis it uses the square root of the residuals. Used to check for the
assumption of homoscedascity. If the line is roughly horizontal and
there is no clear pattern (like a cone) in the scatter plot then
homoscedacity is lokely satisfied.

\begin{Shaded}
\begin{Highlighting}[]
\FunctionTok{plot}\NormalTok{(wH\_lm,}\DecValTok{3}\NormalTok{)}
\end{Highlighting}
\end{Shaded}

\includegraphics{Ass1_files/figure-latex/E3_b_3-1.pdf}

\subsubsection{Residuals vs Leverage}\label{residuals-vs-leverage}

Allows us to identify influential observations \textbf{Leverage:} extent
to which the coefficients in the regression model would change if a
particular observations was removed from the dataset (outliers).
\textbf{Standarized residuals:} standardized difference between a
predicted value for an observation and the actual value of the
observation.

\begin{Shaded}
\begin{Highlighting}[]
\FunctionTok{plot}\NormalTok{(wH\_lm,}\DecValTok{5}\NormalTok{)}
\end{Highlighting}
\end{Shaded}

\includegraphics{Ass1_files/figure-latex/E3_b_4-1.pdf}

\subsection{Task c}\label{task-c-2}

\begin{Shaded}
\begin{Highlighting}[]
\NormalTok{data\_gen }\OtherTok{\textless{}{-}} \ControlFlowTok{function}\NormalTok{(}\AttributeTok{c=}\DecValTok{0}\NormalTok{, }\AttributeTok{t=}\DecValTok{0}\NormalTok{, }\AttributeTok{cm=}\DecValTok{0}\NormalTok{, }\AttributeTok{cp=}\DecValTok{100}\NormalTok{, }\AttributeTok{cs=}\DecValTok{0}\NormalTok{, }\AttributeTok{nnm=}\DecValTok{1}\NormalTok{, }\AttributeTok{nnp=}\DecValTok{10000}\NormalTok{, }\AttributeTok{snn=}\DecValTok{500}\NormalTok{, }\AttributeTok{nnnm=}\DecValTok{0}\NormalTok{, }\AttributeTok{nnnp=}\DecValTok{10000}\NormalTok{, }\AttributeTok{snnn=}\DecValTok{500}\NormalTok{) \{}
  \FunctionTok{set.seed}\NormalTok{(}\DecValTok{42}\NormalTok{)}
  
\NormalTok{  n }\OtherTok{\textless{}{-}} \DecValTok{1000}
\NormalTok{  x }\OtherTok{\textless{}{-}} \DecValTok{1}\SpecialCharTok{:}\NormalTok{n}
  
  \CommentTok{\# Changeable parameters}
  \CommentTok{\# {-} Change the parameters to affect the generated data points below.}
  \CommentTok{\# {-} You may copy this code multiple times to answer all the questions in the exercise.}
  \CommentTok{\# {-} You may find it reasonable to argue for multiple violations from a single generated set of data points.}
  
\NormalTok{  contant }\OtherTok{\textless{}{-}}\NormalTok{ c }
\NormalTok{  trend }\OtherTok{\textless{}{-}}\NormalTok{ t}
\NormalTok{  curve\_magnitue }\OtherTok{\textless{}{-}}\NormalTok{ cm}
\NormalTok{  curve\_period }\OtherTok{\textless{}{-}}\NormalTok{ cp}
\NormalTok{  curve\_shift }\OtherTok{\textless{}{-}}\NormalTok{ cs}
\NormalTok{  normal\_noise\_magnitue }\OtherTok{\textless{}{-}}\NormalTok{ nnm}
\NormalTok{  norm\_noise\_periode }\OtherTok{\textless{}{-}}\NormalTok{ nnp}
\NormalTok{  shift\_norm\_noise }\OtherTok{\textless{}{-}}\NormalTok{ snn}
\NormalTok{  non\_normal\_noise\_magnitue }\OtherTok{\textless{}{-}}\NormalTok{ nnnm}
\NormalTok{  non\_norm\_noice\_periode }\OtherTok{\textless{}{-}}\NormalTok{ nnnp}
\NormalTok{  shift\_non\_norm\_noise }\OtherTok{\textless{}{-}}\NormalTok{ snnn}
  
\NormalTok{  y.gen }\OtherTok{\textless{}{-}}\NormalTok{ contant }\SpecialCharTok{+}
\NormalTok{    trend }\SpecialCharTok{*}\NormalTok{ x }\SpecialCharTok{+} 
\NormalTok{    curve\_magnitue}\SpecialCharTok{*} \FunctionTok{sin}\NormalTok{(}
\NormalTok{      (x}\SpecialCharTok{/}\NormalTok{curve\_period }\SpecialCharTok{+}\NormalTok{ curve\_shift)}\SpecialCharTok{*}\NormalTok{pi}
\NormalTok{      ) }\SpecialCharTok{+} 
\NormalTok{    normal\_noise\_magnitue}\SpecialCharTok{*}\FunctionTok{cos}\NormalTok{(}
\NormalTok{      (x}\SpecialCharTok{/}\NormalTok{norm\_noise\_periode }\SpecialCharTok{+}\NormalTok{ shift\_norm\_noise}\SpecialCharTok{/}\NormalTok{norm\_noise\_periode)}\SpecialCharTok{*}\NormalTok{pi}
\NormalTok{      )}\SpecialCharTok{*}\FunctionTok{rnorm}\NormalTok{(n, }\AttributeTok{sd =} \DecValTok{3}\NormalTok{) }\SpecialCharTok{+}
\NormalTok{    non\_normal\_noise\_magnitue}\SpecialCharTok{*}\FunctionTok{cos}\NormalTok{(}
\NormalTok{      (x}\SpecialCharTok{/}\NormalTok{non\_norm\_noice\_periode }\SpecialCharTok{+}\NormalTok{ shift\_non\_norm\_noise}\SpecialCharTok{/}\NormalTok{non\_norm\_noice\_periode)}\SpecialCharTok{*}\NormalTok{pi}
\NormalTok{      ) }\SpecialCharTok{*} \FunctionTok{rexp}\NormalTok{(n, }\AttributeTok{rate =} \FloatTok{0.2}\NormalTok{) }
  
\NormalTok{  p }\OtherTok{\textless{}{-}} \FunctionTok{qplot}\NormalTok{(x, y.gen, }\AttributeTok{ylab =} \StringTok{"y"}\NormalTok{) }\SpecialCharTok{+}
    \FunctionTok{geom\_point}\NormalTok{(}\AttributeTok{size =} \FloatTok{0.1}\NormalTok{) }\SpecialCharTok{+}
    \FunctionTok{labs}\NormalTok{(}\AttributeTok{title =} \StringTok{"Data generate for linear regrestion"}\NormalTok{)}
  
  \CommentTok{\# Display the plot}
  \FunctionTok{print}\NormalTok{(p)}
  \FunctionTok{return}\NormalTok{(}\FunctionTok{list}\NormalTok{(}\StringTok{"x"}\OtherTok{=}\NormalTok{x,}\StringTok{"y.gen"}\OtherTok{=}\NormalTok{y.gen))}
\NormalTok{\}}
\end{Highlighting}
\end{Shaded}

\subsubsection{Holds all the
assumptions}\label{holds-all-the-assumptions}

\begin{Shaded}
\begin{Highlighting}[]
\NormalTok{gen\_data }\OtherTok{\textless{}{-}} \FunctionTok{data\_gen}\NormalTok{()}
\end{Highlighting}
\end{Shaded}

\begin{verbatim}
## Warning: `qplot()` was deprecated in ggplot2 3.4.0.
## This warning is displayed once every 8 hours.
## Call `lifecycle::last_lifecycle_warnings()` to see where this warning was
## generated.
\end{verbatim}

\includegraphics{Ass1_files/figure-latex/E3_c_plots-1.pdf}

\begin{Shaded}
\begin{Highlighting}[]
\NormalTok{lm.gen }\OtherTok{\textless{}{-}} \FunctionTok{lm}\NormalTok{(gen\_data}\SpecialCharTok{$}\NormalTok{y.gen }\SpecialCharTok{\textasciitilde{}}\NormalTok{ gen\_data}\SpecialCharTok{$}\NormalTok{x)}
\FunctionTok{plot}\NormalTok{(lm.gen, }\AttributeTok{which =} \DecValTok{1}\NormalTok{)}
\end{Highlighting}
\end{Shaded}

\includegraphics{Ass1_files/figure-latex/E3_c_plots-2.pdf}

\begin{Shaded}
\begin{Highlighting}[]
\FunctionTok{plot}\NormalTok{(lm.gen, }\AttributeTok{which =} \DecValTok{2}\NormalTok{)}
\end{Highlighting}
\end{Shaded}

\includegraphics{Ass1_files/figure-latex/E3_c_plots-3.pdf}

\begin{Shaded}
\begin{Highlighting}[]
\FunctionTok{plot}\NormalTok{(lm.gen, }\AttributeTok{which =} \DecValTok{3}\NormalTok{)}
\end{Highlighting}
\end{Shaded}

\includegraphics{Ass1_files/figure-latex/E3_c_plots-4.pdf}

\begin{Shaded}
\begin{Highlighting}[]
\FunctionTok{plot}\NormalTok{(lm.gen, }\AttributeTok{which =} \DecValTok{5}\NormalTok{)}
\end{Highlighting}
\end{Shaded}

\includegraphics{Ass1_files/figure-latex/E3_c_plots-5.pdf}

\subsubsection{Breaks the assumption of
Homoscedastiicty}\label{breaks-the-assumption-of-homoscedastiicty}

\begin{Shaded}
\begin{Highlighting}[]
\NormalTok{gen\_data }\OtherTok{\textless{}{-}} \FunctionTok{data\_gen}\NormalTok{(}\AttributeTok{nnnm =} \SpecialCharTok{{-}}\DecValTok{10000}\NormalTok{, }\AttributeTok{snnn =} \DecValTok{5000}\NormalTok{, }\AttributeTok{nnnp =} \DecValTok{50000}\NormalTok{)}
\end{Highlighting}
\end{Shaded}

\includegraphics{Ass1_files/figure-latex/E3_c_bHomoscedastiicty-1.pdf}

\begin{Shaded}
\begin{Highlighting}[]
\NormalTok{lm.gen }\OtherTok{\textless{}{-}} \FunctionTok{lm}\NormalTok{(gen\_data}\SpecialCharTok{$}\NormalTok{y.gen }\SpecialCharTok{\textasciitilde{}}\NormalTok{ gen\_data}\SpecialCharTok{$}\NormalTok{x)}
\FunctionTok{plot}\NormalTok{(lm.gen, }\AttributeTok{which =} \DecValTok{2}\NormalTok{)}
\end{Highlighting}
\end{Shaded}

\includegraphics{Ass1_files/figure-latex/E3_c_bHomoscedastiicty-2.pdf}

\subsubsection{Breaks the assumption of
Linearity}\label{breaks-the-assumption-of-linearity}

\begin{Shaded}
\begin{Highlighting}[]
\NormalTok{gen\_data }\OtherTok{\textless{}{-}} \FunctionTok{data\_gen}\NormalTok{(}\AttributeTok{nnnm =} \SpecialCharTok{{-}}\DecValTok{10000}\NormalTok{, }\AttributeTok{snnn =} \DecValTok{5000}\NormalTok{, }\AttributeTok{cs=}\DecValTok{10000}\NormalTok{, }\AttributeTok{cm =} \DecValTok{10000}\NormalTok{, }\AttributeTok{snn =} \DecValTok{10000}\NormalTok{)}
\end{Highlighting}
\end{Shaded}

\includegraphics{Ass1_files/figure-latex/E3_c_bLinear-1.pdf}

\begin{Shaded}
\begin{Highlighting}[]
\NormalTok{lm.gen }\OtherTok{\textless{}{-}} \FunctionTok{lm}\NormalTok{(gen\_data}\SpecialCharTok{$}\NormalTok{y.gen }\SpecialCharTok{\textasciitilde{}}\NormalTok{ (gen\_data}\SpecialCharTok{$}\NormalTok{x)}\SpecialCharTok{\^{}}\DecValTok{4}\NormalTok{)}
\FunctionTok{plot}\NormalTok{(lm.gen, }\DecValTok{1}\NormalTok{)}
\end{Highlighting}
\end{Shaded}

\includegraphics{Ass1_files/figure-latex/E3_c_bLinear-2.pdf}

\subsubsection{Breaks the assumption of
Normality.}\label{breaks-the-assumption-of-normality.}

\begin{Shaded}
\begin{Highlighting}[]
\NormalTok{gen\_data }\OtherTok{\textless{}{-}} \FunctionTok{data\_gen}\NormalTok{(}\AttributeTok{nnnm =} \DecValTok{10000}\NormalTok{, }\AttributeTok{snnn =} \DecValTok{5000}\NormalTok{, }\AttributeTok{cs=}\DecValTok{10000}\NormalTok{, }\AttributeTok{cm =} \DecValTok{10000}\NormalTok{, }\AttributeTok{snn =} \DecValTok{10000}\NormalTok{, }\AttributeTok{nnm =} \DecValTok{0}\NormalTok{, }\AttributeTok{nnnp =} \DecValTok{5000}\NormalTok{, }\AttributeTok{t =} \DecValTok{1}\NormalTok{)}
\end{Highlighting}
\end{Shaded}

\includegraphics{Ass1_files/figure-latex/E3_c_bNorm-1.pdf}

\begin{Shaded}
\begin{Highlighting}[]
\NormalTok{lm.gen }\OtherTok{\textless{}{-}} \FunctionTok{lm}\NormalTok{(gen\_data}\SpecialCharTok{$}\NormalTok{y.gen }\SpecialCharTok{\textasciitilde{}}\NormalTok{ gen\_data}\SpecialCharTok{$}\NormalTok{x)}
\FunctionTok{plot}\NormalTok{(lm.gen, }\DecValTok{2}\NormalTok{)}
\end{Highlighting}
\end{Shaded}

\includegraphics{Ass1_files/figure-latex/E3_c_bNorm-2.pdf}

\subsection{Task d}\label{task-d-1}

\subsubsection{Breaks Homoscedasticity}\label{breaks-homoscedasticity}

The error distibution is not consistent, aka
\[ \epsilon_i \ne \sigma^2 \] instead it is
\[ \sigma_{i}^2 = x_{i} \sigma^2 \], in other words heteroscedastic.

If plot X against Y you will see a cone shape instead of expected line,
clearly there are different distributions of variances based on X

\subsubsection{Breaks Linearity}\label{breaks-linearity}

The Residual vs Fitted shows a clear cone as well as a clear non-linear
pattern between X and Y

\subsubsection{Breaks Normality}\label{breaks-normality}

The Q-Q Normal plot clearly shows high bias towards the tails

\section{Exercise 4 - correlation and partial
correlation}\label{exercise-4---correlation-and-partial-correlation}

\subsection{Task a}\label{task-a-3}

\textbf{Correlation:} Degree to which a pair of variables/factors are
linearly related. As in the increase in one variable either increases or
decreases another. Correlation does not imply causation, even if Smoking
and higher life expectancy is higher in one area, that doesn't
automatically imply that smoking is healthy. The correlation
coefficient, typically denoted as ρ, ranges from -1 to 1: * A value of 1
indicates a perfect positive correlation, * A value of -1 indicates a
perfect negative correlation, * A value of 0 indicates no correlation.

\textbf{Effect of Scaling:} When variables are scaled by constants α1
and α2, the correlation is unaffected in magnitude but may change sign:
ρW1W2 = cov(α1X, α2Y) / (sqrt(var(α1X)) * sqrt(var(α2Y))) This
simplifies to: ρW1W2 = sgn(α1α2) * ρXY Thus, scaling does not change the
inherent linear relationship but can alter its direction.

Higher variance decreases correlation, while higher covariance increases
it. In other words the more the two variables/factors trend and the less
variance from that trend they have the more correlated the two
varaibles/factors are.

\subsection{Task b}\label{task-b-3}

\textbf{Partial correlation:} measures the degree of association between
two random variables, with the effect of a set of controlling random
variables removed.
\href{https://en.wikipedia.org/wiki/Partial_correlation}{Wiki} Helps to
understand the relationship between two variables, controlling for the
effect of one or more additional variables.

The formula describes the correlation between the residuals \(e_X\) and
\(e_Y\) resulting from the linear regression of X with Z and of Y with
Z, respectively.

\subsubsection{Scenarios}\label{scenarios}

\begin{enumerate}
\def\labelenumi{\arabic{enumi}.}
\tightlist
\item
  You suspect there are inter dependencies between X=Study hours per
  week, Y=Exam score and Z=IQ
\item
  You suspect there are inter dependencies between X=Calories consumed,
  Y=Weight loss and Z=Hours of intense exercise.
\item
  You suspect there are inter dependencies between X=Time spent on
  social media, Y=Self-esteem and Z=Age
\end{enumerate}

\subsection{Task c}\label{task-c-3}

Yes, the property of correlation regarding scaling holds for partial
correlation as well. Just like regular correlation, partial correlation
is not affected by scaling of the variables, only the direction or sign
can change. Let's do the math so see how this is the case:

\begin{enumerate}
\def\labelenumi{\arabic{enumi}.}
\item
  In the partial correlation formula we have: Pxy Pxz Pyz
\item
  if we then scale the x and y variables with a1 and a2 we get: Pxy =
  sgn(a1,a2)\emph{Pxy Pxz = sgn(a1)}Pxz Pyz = sgn(a2)*Pyz
\item
  Then we substitute the coefficients with the scaled ones and obtain
  the formula: Pxy\textbar z = (Pxy - Pxz\emph{Pyz) /
  (sqrt(1-(Pxz)\^{}2)}sqrt(1-(Pyz)\^{}2) -\textgreater{}
  (sgn(a1,a2)\emph{Pxy - sgn(a1)}Pxz\emph{sgn(a2)}Pyz) /
  (sqrt(1-(sgn(a1)*Pxz)\textsuperscript{2)\emph{sqrt(1-(sgn(a2)}Pyz)}2)
\item
  after doing the math and simplifying the formula we obtain
  sgn(a1,a2)*Pxy\textbar z
\end{enumerate}

\subsection{Task d}\label{task-d-2}

\begin{Shaded}
\begin{Highlighting}[]
\NormalTok{weatherHistory }\OtherTok{\textless{}{-}} \FunctionTok{read.csv}\NormalTok{(}\StringTok{"weatherHistory.csv"}\NormalTok{)}
\NormalTok{select\_wH }\OtherTok{\textless{}{-}}\NormalTok{ weatherHistory }\SpecialCharTok{\%\textgreater{}\%}\NormalTok{ dplyr}\SpecialCharTok{::}\FunctionTok{select}\NormalTok{(Temperature..C., Apparent.Temperature..C., Humidity)}
\end{Highlighting}
\end{Shaded}

\begin{Shaded}
\begin{Highlighting}[]
\NormalTok{pairwise\_corr }\OtherTok{\textless{}{-}}\NormalTok{ select\_wH }\SpecialCharTok{\%\textgreater{}\%}
  \FunctionTok{cor}\NormalTok{() }\SpecialCharTok{\%\textgreater{}\%}
  \FunctionTok{data.frame}\NormalTok{()}
\NormalTok{pairwise\_corr}
\end{Highlighting}
\end{Shaded}

\begin{verbatim}
##                          Temperature..C. Apparent.Temperature..C.   Humidity
## Temperature..C.                1.0000000                0.9926286 -0.6322547
## Apparent.Temperature..C.       0.9926286                1.0000000 -0.6025710
## Humidity                      -0.6322547               -0.6025710  1.0000000
\end{verbatim}

\begin{Shaded}
\begin{Highlighting}[]
\NormalTok{part\_corr }\OtherTok{\textless{}{-}}\NormalTok{ select\_wH }\SpecialCharTok{\%\textgreater{}\%}
  \FunctionTok{pcor}\NormalTok{()}

\NormalTok{part\_corr}
\end{Highlighting}
\end{Shaded}

\begin{verbatim}
## $estimate
##                          Temperature..C. Apparent.Temperature..C.   Humidity
## Temperature..C.                1.0000000                0.9892298 -0.3528185
## Apparent.Temperature..C.       0.9892298                1.0000000  0.2664916
## Humidity                      -0.3528185                0.2664916  1.0000000
## 
## $p.value
##                          Temperature..C. Apparent.Temperature..C. Humidity
## Temperature..C.                        0                        0        0
## Apparent.Temperature..C.               0                        0        0
## Humidity                               0                        0        0
## 
## $statistic
##                          Temperature..C. Apparent.Temperature..C.   Humidity
## Temperature..C.                   0.0000               2098.90993 -117.10342
## Apparent.Temperature..C.       2098.9099                  0.00000   85.86793
## Humidity                       -117.1034                 85.86793    0.00000
## 
## $n
## [1] 96453
## 
## $gp
## [1] 1
## 
## $method
## [1] "pearson"
\end{verbatim}

\begin{itemize}
\tightlist
\item
  Apparent.Temperature..C. and Temperature..C. has a partial correlation
  value of ca 0.99 which implies that they are highly consistent with
  each other and increase with each other. Very low difference between
  pairwise to partial where they are equal if rounded to nearest 2
  decimal, which implies that Z=Humidity has very little effect on them.
\item
  Humidity and Temperature..C. has a low partial correlation value of
  -0.35 which implies that they are not consistent with each other and
  decreases when the other increases. Decreases from -0.63 to -0.35
  which implies that Z=Apparent.Temperature.C.. is affecting them. Z is
  a significant confounder.
\item
  Humidity and Apparent.Temperature..C. has a low partial correlation
  value of 0.27 which implies that they are not consistent with each
  other and increase with each other. Dropped from pairwise correlation
  -0.60 to 0.27 which implies that Z=Temperature..C. is significantly
  affecting them. Z is a significant cofunder.
\end{itemize}

\end{document}
